\section{Introduction}

Thermodynamics developed rapidly in the mid-19th century through contributions from figures such as Carnot, Joule, Clausius, and Kelvin, who built the theoretical framework for explaining the macroscopic behavior of physical systems. Meanwhile, kinetic gas theory and the concept of entropy developed by Boltzmann connected molecular dynamics with entropy, forming the foundation for the birth of statistical mechanics. This discipline not only explains macroscopic order but also spontaneous fluctuations that occur naturally in physical systems \cite{Pathria2001}.

Research on the thermodynamic properties of magnetic materials has been central to condensed matter physics for several decades, not only because of the richness of physical phenomena it offers but also because of its potential applications in technology, from data storage to computational materials. Understanding how microscopic spin interactions (magnetic moments) influence macroscopic system behavior when exposed to temperature or magnetic field changes is the primary goal of this study.

Phase transitions represent one of the most fascinating aspects of statistical physics, where changes in macroscopic parameters such as temperature or pressure can trigger drastic changes in the collective structure and properties of a system, for example from paramagnetic phase (no magnetic order) to ferromagnetic (long-range magnetic order) in magnetic materials or vice versa. Studies of phase transitions not only reveal fundamental material properties but also provide insights into the universality of system behavior near critical points, regardless of microscopic details \cite{Tokura2019}.

In general, phase transitions are related to the phenomenon of system symmetry breaking. For phase changes caused by thermal fluctuations, the system is at a high degree at high temperatures because all configuration spaces are allowed. Decreasing temperature will reduce thermal fluctuations and result in properties being in a stable state \cite{Surungan2017}.

One important phenomenon in phase changes is the occurrence of spontaneous magnetization. In ferromagnetic systems, when the temperature is lowered until reaching a certain temperature called the critical temperature, spontaneous magnetization will occur. The system undergoes a phase change from a paramagnetic system to a ferromagnetic system. The critical temperature $T_c$ for ferromagnetic systems is called the Curie temperature. This phenomenon is very interesting to study because it involves spin interactions, which are microscopic factors.

Ernest Ising (1925) introduced a model that can explain the spontaneous magnetization phenomenon of ferromagnetic (FM) systems. The Ising model is a simple form of the Heisenberg model to solve the model proposed by Lenz (1920) in studying FM phase changes at Curie temperature. This model contains discrete variables representing the magnetic moment of atomic spins with values $s = \pm 1$. These spins are modeled in a lattice where each spin can interact with its nearest neighbors.

The 2D Ising model is one example of a simple statistical model to show spontaneous magnetization in a system. In this study, we will examine the cubic spin model, which is one of the discrete spin models with polyhedral symmetry. Polyhedral symmetry for spin models is obtained by dividing the 4π solid angle equally from the sphere structure. There are five possible types of models from polyhedral structures: tetrahedron, octahedron, hexahedron, icosahedron, and dodecahedron.

These spins are modeled in a lattice that interacts with each other like the Ising model. Using the Monte Carlo simulation method, the order parameter can be calculated and the critical temperature of each model can be estimated. For this research, it is specifically focused on studying the vertex-cubic spin model.

In previous research \cite{Surungan2008}, FM systems on 2D lattices were studied, and other research \cite{Sutiono2013} studied systems with different lattice structures, namely layered lattices (3D). Systems with larger spatial dimensions are theoretically expected to have higher critical temperatures because the number of neighbors from each spin is greater.

This research specifically focuses on studying the thermodynamic properties of magnetic models with vertex-cubic spin models arranged on lattices of size $L \times L \times n$. Spin interactions are defined through scalar products of spin vectors, similar to the Heisenberg model discretized on 8 vertex-cubic orientations. This research aims to investigate how phase transitions, magnetization, and other thermodynamic properties such as specific heat are influenced by lattice topology and the potential frustration it creates. The results are expected to provide further understanding of magnetic behavior that not only enriches statistical physics theory but also has potential relevance for the development of new functional magnetic materials in the future. 