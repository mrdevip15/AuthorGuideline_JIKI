\section{Conclusion}

Research on the thermodynamic properties of cubic magnetic models on quasi-three-dimensional lattices has been successfully conducted. This model adopts spin orientations pointing to one of the 8 vertex-cubic vector directions, which is a discrete representation of the Heisenberg model and belongs to the polyhedral symmetry category.

Through Monte Carlo simulation with Wolff algorithm implementation, the thermodynamic properties of the system have been carefully observed. The simulation results show that:

\begin{enumerate}
\item Increasing lattice size sharpens the phase transition characteristics of the system. Specific heat peaks become sharper, magnetization decay becomes clearer, and internal energy changes become steeper. This indicates that larger systems represent thermodynamic behavior more accurately.

\item Increasing the number of layers produces more explicit and stable phase transitions. The more layers stacked, the higher the critical temperature achieved, and the stronger the spin order maintained until the transition temperature is reached. This shows that interlayer coupling plays an important role in strengthening ordered phases.

\item Critical temperatures are identified consistently through several observable factors, such as specific heat peaks, minimum Binder Cumulant, and magnetization decay. The trends obtained show that critical temperature increases with increasing lattice size and number of layers, and is more stable in large systems with thick layers.

\item The ratio of root-mean-square magnetization and magnetic moment shows that spin order is more stable in layered systems, even at temperatures approaching the transition. Binder Cumulant intersections become sharper with increasing $L$ and $n$, indicating stronger and clearer critical behavior.

\item The layered vertex-cubic spin model provides a relevant description of intermediate-dimensional physical systems, such as thin magnetic films or other layered structures. This approach has proven effective for studying phase transitions and critical phenomena in systems with discrete orientations and limited spatial interactions.
\end{enumerate}

\section*{Acknowledgement}

The authors would like to thank the research community for their valuable contributions to the field of statistical physics and magnetic materials. Special thanks to the reviewers for their constructive feedback. 